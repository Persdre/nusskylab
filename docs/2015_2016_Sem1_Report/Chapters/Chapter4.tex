\chapter{Security} \label{security}

During the development of Skylab, security has always been one of the priorities. Common attacks such as SQL injection, XSS attacks and CSRF attacks have been partly addressed by Rails framework itself and Skylab's main focus is to make sure authentication process and access control are well-defined and can withstand various attacks.

\section{User authentication}

There are basically 2 ways to login to Skylab: email and password combination and NUS OpenID. For email and password combination we are using Devise to manage authentication related information in \textit{User} model. Devise has been used widely in Rails community and currently there are more than 13,000 stars of Devise repository on GitHub\cite{citation13}. What is more, according to security scanning of Devise by Hakiri there is no security warning, which further proves the trust-ability of Devise\cite{citation13}. As for NUS OpenID, it is using OAuth 2.0 as authentication protocol, which is believed to relatively safe and currently is adopted by many OpenID providers as well\cite{citation14}.

\section{Access control}

Skylab is adopting role based access control strategy to grant different permissions for different users. There are basically 3 levels of checking for each incoming requests:

\begin{itemize}
  \item \textbf{login\_required:} if a action is login\_required then a user must sign to be granted for the access.
  \item \textbf{admin\_only:} if a action is admin\_only then only users with role of \textit{Admin} can access.
  \item \textbf{customized:} if a action requires login and is not only accessed by admins then the current logged-in user will be checked against the list of users who have the permissions.
\end{itemize}

As this checking procedure is a common to nearly all actions, it is defined in \textit{ApplicationController} which will be effectively inherited by all controllers in Skylab. And further customization can be done by overriding as well.
