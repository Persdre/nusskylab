\chapter{Public Views} \label{publicprofile}

Before Skylab was implemented, the display of past Orbital staff and projects was done by means of a manually curated WordPress post.   Project links for teams and team members were all manually created, which was a tedious and error-prone process. Since Skylab already captures all information on teams and students, publicly-accessible (i.e., without needing to login) pages are a logical form of data export that has been created to supersede these past facilities.

\section{User Avatars} \label{useravatar}

Skylab does not currently handle image uploading, as it requires a lot more storage and more efforts in security as well. However, we need to support user avatar pictures as this enhances the user experience of Skylab.  To implement this, we have integrated with an external third party service.   We have decided to use Gravatar, a globally recognized avatar for any user\cite{citationgravatar}. Gravatar is popularized by Wordpress and in use by many sites.  It is easy to integrate and a number of users already use gravatar for creating a universal user avatar for themselves.  The steps to get an avatar are as follows:

\begin{itemize}
  \item Trim the user's email address, convert it to lower case, and calculate MD5 hash of the transformed email address.
  \item Use the hash computed in previous step and append it to \url{http://www.gravatar.com/avatar/} to obtain the gravatar's image source.
\end{itemize}

Gravatar serves our purpose of serving users' avatars while bypassing the complication of supporting image uploading and storage, as this is handled by the service. And the most important part in arriving at this solution is that it is a relatively quick, easy yet clean to serve users' profile images. Compromises like this happen as we need to weigh advantages and disadvantages of different approaches and take the time taken and priority of issues in implementation into consideration. In the future, if we find too many inconveniences from using Gravatar, we may decide to host avatar images on our own server, and handle problems arising from the support of file uploading and storage.
